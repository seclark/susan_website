%%%%%%%%%%%%%%%%%%%%%%%%%%%%%%%%%%%%%%%%%
% Medium Length Professional CV
% LaTeX Template
% Version 2.0 (8/5/13)
%
% This template has been downloaded from:
% http://www.LaTeXTemplates.com
%
% Original author:
% Trey Hunner (http://www.treyhunner.com/)
%
% Important note:
% This template requires the resume.cls file to be in the same directory as the
% .tex file. The resume.cls file provides the resume style used for structuring the
% document.
%
%%%%%%%%%%%%%%%%%%%%%%%%%%%%%%%%%%%%%%%%%

%----------------------------------------------------------------------------------------
%	PACKAGES AND OTHER DOCUMENT CONFIGURATIONS
%----------------------------------------------------------------------------------------

\documentclass{resume_clark} % Use the custom resume.cls style

\usepackage[left=0.75in,top=0.6in,right=0.75in,bottom=0.6in]{geometry} % Document margins
\usepackage[dvipsnames]{xcolor}
\usepackage{hyperref}
\hypersetup{colorlinks,linkcolor=,urlcolor=NavyBlue}
\usepackage{etaremune}

\name{\sc Susan E. Clark} % Your name
\cvheading{\textit{Curriculum Vitae}}
\address{1411 Pupin Hall \hfill seclark@astro.columbia.edu \\ 550 West 120th Street \hfill \href{http://www.astro.columbia.edu/~seclark}{\tt astro.columbia.edu/$\sim$seclark} \\ New York, New York 10027 \hfill  github: seclark\\} % Your address
\lhsaddress{stuff}

\newcommand{\hi}{H{\sc i}~}
\newcommand{\HI}{H{\sc i}}

\begin{document}

%----------------------------------------------------------------------------------------
%	EDUCATION SECTION
%----------------------------------------------------------------------------------------

\begin{rSection}{Education}

{\bf Columbia University}  \\ 
Ph.D. Candidate, Astrophysics  \hfill {2017 (expected)}  \\
\hspace*{0.5cm}Thesis Project: {\em Magnetic Fields in the Interstellar Medium}\\
\hspace*{0.5cm}Thesis Advisors: Mary E. Putman, Joshua E.G. Peek\\
M.A., M.Phil, Astrophysics \hfill {2014}  \\

\vspace{-0.3cm}

{\bf The University of North Carolina at Chapel Hill}\\ 
B.S., Physics \hfill {2012}\\

\vspace{-0.3cm}

{\bf Thomas Jefferson High School for Science \& Technology} \hfill {2008}\\

\end{rSection}

%----------------------------------------------------------------------------------------
%	FELLOWSHIPS, HONORS AND AWARDS
%----------------------------------------------------------------------------------------

\begin{rSection}{Honors, Awards and Grants}
NSF Graduate Research Fellowship\hfill {2012 -- present}\\
Columbia Dean's Fellowship\hfill {2012 -- present}\\
ASNY Graduate Student Paper Prize \hfill{2016}\\
CCAPP Price Prize in Cosmology and AstroParticle Physics \hfill{2016}\\
PI, VLA Observing Proposal 16A-133, 8 hours \hfill{2016}\\
PRL Editors' Recommendation Paper\hfill {2015}\\
Morehead-Cain Scholarship\hfill {2008 -- 2012}\\
\hspace*{0.5cm}\href{http://www.moreheadcain.org}{Full scholarship} to UNC-Chapel Hill\\

\end{rSection}
%------------------------------------------------

%\begin{rSubsection}{TinySoft}{January 2008 - April 2010}{Web Designer \& Developer}{Gainesville, GA}
%\item Vivamus PostgreSQL fermentum semper porta. Nunc diam velit PHP, adipiscing ut tristique vitae
%\item Maecenas convallis ullamcorper ultricies stylesheets.
%\item Quisque mi metus, unit tests CSS ornare sit amet fermentum et, tincidunt et orci.
%\item Curabitur venenatis pulvinar tellus gravida ornare. Sed et erat faucibus nunc euismod ultricies ut id
%\end{rSubsection}


%----------------------------------------------------------------------------------------
%	REFEREED PUBLICATIONS
%----------------------------------------------------------------------------------------

\begin{rSection}{Publications}

%8 journal articles, 4 first author
First-author journal articles

\begin{etaremune}

\item {\bf S.E. Clark} \& J.S. Oishi. \textit{The weakly nonlinear magnetorotational instability in a global, cylindrical Taylor-Couette flow}. 2016, submitted to ApJ. \href{https://arxiv.org/abs/1610.01603}{\tt arXiv:1610.01603}

\item {\bf S.E. Clark} \& J.S. Oishi. \textit{The weakly nonlinear magnetorotational instability in a thin-gap Taylor-Couette flow}. 2016, submitted to ApJ. \href{http://arxiv.org/abs/1610.01616}{\tt arXiv:1610.01616}

\item {\bf S.E. Clark}, J. Colin Hill, J.E.G. Peek, M.E. Putman, B.L. Babler. \textit{Neutral hydrogen structures trace dust polarization angle: Implications for cosmic microwave background foregrounds}. 2015, PRL 115, 241302. Selected as PRL Editors' Recommendation. \href{http://arxiv.org/abs/1508.07005}{\tt arXiv:1508.07705}. 

\item {\bf S.E. Clark}, J.E.G. Peek, M.E. Putman. {\em Magnetically aligned {\textit{\textsc{Hi}}} fibers and the Rolling Hough Transform}. 2014, {ApJ, 789, 82}. \href{http://arxiv.org/abs/1312.1338}{\tt arXiv:1312.1338}
\end{etaremune}

Other journal articles

\begin{etaremune}
\item F. Heitsch, B. Bartell, {\bf S.E. Clark}, J.E.G. Peek, D. Cheng, M.E. Putman. \textit{Three-dimensional orientation of compact high velocity clouds}. 2016, MNRAS Letters 462, L46. \href{http://arxiv.org/abs/1606.06689}{\tt arXiv:1606.06689}.

\item J. Malinen, L. Montier, J. Montillaud, M. Juvela, I. Ristorcelli, {\bf S.E. Clark}, O. Bern\'e, J.-Ph. Bernard, V.-M. Pelkonen, D.C. Collins. \textit{Matching dust emission structures and magnetic field in high-latitude cloud L1642: comparing Herschel and Planck maps}. 2016, MNRAS 460, 1934. \href{http://arxiv.org/abs/1512.03775}{\tt arXiv:1512.03775}. 

\item N.M. McClure-Griffiths, S. Stanimirovi\' c, [5 authors], {\bf S.E. Clark}, [3 authors]. \textit{Galactic and Magellanic evolution with the SKA}. 2015, from ``Advancing Astrophysics with the Square Kilometre Array", PoS. \href{http://arxiv.org/abs/1501.01130}{\tt arXiv:1501.01130}

\item W.-H. Hsu, M.E. Putman, F. Heitsch, S. Stanimirovi\' c, J.E.G. Peek, {\bf S.E. Clark}. {\em Physical properties of Complex C halo clouds}. 2011, {AJ, 141, 57} \href{https://arxiv.org/abs/1011.0011}{\tt arXiv:1011.0011}
\end{etaremune}

Conference proceedings

\begin{etaremune}
\item {\bf S.E. Clark}, J.E.G. Peek, J. Colin Hill, M.E. Putman. \textit{Quantifying the magnetic alignment of {\textit{\textsc{Hi}}} and dust in the diffuse ISM.} 2016, In P. Jablonka, Ph. Andr\'e, F. van der Tak (Eds.) {\it From Interstellar Clouds to Star-forming Galaxies: Universal Processes?} Proceedings of the International Astronomical Union Symposia and Colloquia, IAU 315, Honolulu, Hawaii
\end{etaremune}

\end{rSection}

%----------------------------------------------------------------------------------------
%	TALKS
%----------------------------------------------------------------------------------------
\begin{rSection}{Scientific Talks}
26 presentations, 10 invited talks/colloquia 
%\textbf{Invited Talks and Colloquia}
\begin{etaremune}

%\end{etaremune}

%\textbf{Contributed Talks and Posters}

%\begin{etaremune}

\item Invited Talk, OSU CCAPP Price Prize Seminar, Columbus, Ohio \hfill{Sept. 2016}

\item Invited Talk, Harvard-Smithsonian CfA Galaxies \& Cosmology Seminar, \hfill{Sept. 2016}\\
Cambridge, Massachussetts

\item Invited Talk, CITA Seminar, Toronto, Canada \hfill {Aug. 2016}

%\item{\textit{Properties of Diffuse Atomic Fibers}\hfill {May 2016}\\ Talk, Star Formation, Magnetic Fields, and Diffuse Matter in the Galaxy, Madison, Wisconsin}
\item Talk, Star Formation, Magnetic Fields, and Diffuse Matter in the Galaxy, \hfill{May 2016}\\
Madison, Wisconsin 

%\item{\textit{Unique Approaches to Magnetic Phenomena: Polarized CMB Foregrounds and the Magnetorotational Instability}\hfill {Feb. 2016}\\ Talk, JILA Seminar, Colorado University Boulder, Boulder, Colorado}
\item Invited Talk, JILA Seminar, Colorado University Boulder, Boulder, Colorado \hfill {Feb. 2016}

%\item{\textit{Measuring B-Mode Polarization Foregrounds with Neutral Hydrogen}\hfill {Jan. 2016}\\ AMNH Colloquium, New York, New York}
\item Invited Talk, AMNH Colloquium, New York, New York \hfill {Jan. 2016}

%\item{\textit{Measuring B-Mode Polarization Foregrounds with Neutral Hydrogen}\hfill {Nov. 2015}\\ Brown Astrophysics Seminar, Providence, Rhode Island }
\item Invited Talk, Brown Astrophysics Seminar, Providence, Rhode Island \hfill {Nov. 2015}

%\item{\textit{Measuring B-mode Polarization Foregrounds with Neutral Hydrogen}\hfill{Oct. 2015}\\ Institut de Recherche en Astrophysique et Plan\'etologie ISM Journal Club, Toulouse, France}
\item Invited Talk, IRAP ISM Seminar, Toulouse, France \hfill{Oct. 2015}

%\item{\textit{Measuring B-mode Polarization Foregrounds with Neutral Hydrogen}\hfill {Oct. 2015}\\ Talk, Magnetic Fields in the Universe V, Corsica, France}
\item Talk, Magnetic Fields in the Universe V, Corsica, France \hfill {Oct. 2015}

%\item{\textit{Magnetically Aligned HI and \textit{Planck} Polarized Dust}\hfill {Oct. 2015}\\ IAS, ENS, and CEA/Saclay Joint ISM Seminar, IAS, Orsay, France }
\item Invited Talk, IAS, ENS, and CEA/Saclay Joint ISM Seminar, IAS, Orsay, France \hfill {Oct. 2015}

%\item{\textit{HI structures trace dust polarization angle: Implications for CMB foregrounds}\hfill {Sept. 2015}\\ Talk, Experimental CMB Journal Club, Columbia University, New York, NY}
\item Talk, Experimental CMB Journal Club, Columbia University, New York, NY \hfill {Sept. 2015}

%\item{\textit{HI Shape Traces Planck Dust Polarization: An Independent Determination of Polarized CMB Foregrounds}\hfill {Aug. 2015}\\ Talk, IAU Focus Meeting 5, The Legacy of Planck, Honolulu, Hawai'i}
\item Talk, IAU Focus Meeting 5, The Legacy of Planck, Honolulu, Hawaii \hfill {Aug. 2015}

%\item{\textit{Magnetically Aligned \textit{HI} and Dust: Measuring the Physical Properties of \hi Fibers}\hfill {Aug. 2015}\\ Poster, IAU Symposium 315, From Interstellar Clouds to Star-Forming Galaxies: Universal Processes?, Honolulu, Hawai'i}
\item Poster, IAU Symposium, From Interstellar Clouds to Star-Forming Galaxies: \hfill {Aug. 2015}\\Universal Processes?, Honolulu, Hawaii

%\item{\textit{Measuring B-mode Polarization Foregrounds with Neutral Hydrogen}\hfill {May 2015}\\ Talk, Pontifica Universidad Catolica, Santiago, Chile}
\item Talk, Pontifica Universidad Cat\'olica, Santiago, Chile \hfill {May 2015}

%\item{\textit{Magnetically Aligned \textit{HI} and Dust in the ISM}\hfill {May 2015}\\ Talk, Midwest Magnetic Fields Workshop, Madison, Wisconsin}
\item Talk, Midwest Magnetic Fields Workshop, Madison, Wisconsin \hfill {May 2015}

%\item{\textit{The Saturation of the Magnetorotational Instability via Weakly Nonlinear Analysis}\hfill {Dec. 2014}\\ Princeton Plasma Physics Laboratory Theory Seminar, Princeton, New Jersey }
\item Invited Talk, PPPL Theory Seminar, Princeton, New Jersey \hfill {Dec. 2014}

%\item{\textit{Magnetically Aligned HI Fibers}\hfill{Oct. 2014}\\ Poster, NSF Directorate for Mathematical \& Physical Sciences, New York, New York}
\item Poster, NSF Directorate for Mathematical \& Physical Sciences, New York, New York \hfill {Oct. 2014}

%\item{\textit{Quantifying Linear Structure with the Rolling Hough Transform and the dRHT}\hfill{Oct. 2014}\\ Poster, Filamentary Structure in Molecular Clouds, NRAO, Charlottesville, Virginia}
\item Poster, Filamentary Structure in Molecular Clouds, NRAO, Charlottesville, Virginia \hfill {Oct. 2014}

%\item{\textit{Exploring MRI Saturation via Weakly Nonlinear Analysis}\hfill {Aug. 2014}\\ Conference: Non-Ideal MHD, Stability, and Dissipation in PPDs, Copenhagen, Denmark}
\item Invited Talk, Non-Ideal MHD, Stability, and Dissipation in PPDs, \hfill {Aug. 2014} \\ Copenhagen, Denmark 

%\item{\textit{Magnetically Aligned HI Fibers}\hfill{May 2014}\\ Poster, Galactic Science with the SKA and its Pathfinders, Leiden, Netherlands}
\item Poster, Galactic Science with the SKA and its Pathfinders, Leiden, Netherlands \hfill {May 2014}

%\item{\textit{Magnetically Aligned HI Fibers}\hfill{April 2014}\\ Talk, Midwest Magnetic Fields Workshop, Madison, Wisconsin}
\item Talk, Midwest Magnetic Fields Workshop, Madison, Wisconsin \hfill {April 2014}

%\item{\textit{Magnetized HI Fibers and the Rolling Hough Transform}\hfill{July 2013}\\ Talk, Phases of the ISM Conference, Heidelberg, Germany}
\item Talk, Phases of the ISM Conference, Heidelberg, Germany \hfill {July 2013}

%\item{\textit{The Disruption of High-Velocity Clouds in the Milky Way}\hfill{April 2012}\\ Talk, Senior Research Symposium, Chapel Hill, North Carolina}
\item Talk, Senior Research Symposium, Chapel Hill, North Carolina \hfill {April 2012}

%\item{\textit{Dust-to-Gas Comparisons with GALFA-HI}\hfill{Jan. 2012}\\ Poster, AAS Winter Meeting, Austin, Texas}
\item Poster, AAS Winter Meeting, Austin, Texas \hfill {Jan. 2012}

%\item{\textit{Gas/Dust Comparisons Using GALFA-HI Data}\hfill{Aug. 2011}\\ Talk, GALFA-HI Collaboration Meeting, Madison, Wisconsin}
\item Talk, GALFA-HI Collaboration Meeting, Madison, Wisconsin \hfill {Aug. 2011}

%\item{\textit{Dust and Gas in the Interstellar Medium}\hfill{Aug. 2011}\\ Talk, REU Symposium, Arecibo Observatory, Puerto Rico}
\item Talk, REU Symposium, Arecibo Observatory, Puerto Rico \hfill {Aug. 2011}

\end{etaremune}

%\textbf{Public Talks and Guest Lectures}

%\item{\textit{The Dusty, Magnetized Interstellar Medium and CMB Polarization} \hfill{March 30, 2016} \\ Hunter College ``Unsolved Problems in Astrophysics" course}

\end{rSection}

%----------------------------------------------------------------------------------------
%	TELESCOPE TIME
%----------------------------------------------------------------------------------------

%\begin{rSection}{Competitively Obtained Telescope Time}
%PI, 

%\end{rSection}

%----------------------------------------------------------------------------------------
%	TEACHING EXPERIENCE
%----------------------------------------------------------------------------------------

\begin{rSection}{Teaching Experience}

Head Teaching Assistant, Columbia Department of Astronomy \hfill {2015 -- 2016}\\
Guest Lecturer, Hunter College Unsolved Problems in Astrophysics course \hfill {2016} \\
Grader, Columbia University, Life in the Universe \hfill {2014}\\
Instructor, Columbia University, Observational Astronomy Lab \hfill {2014}\\
Instructor, Columbia University, Earth, Moon, \& Planets Lab \hfill {2013}\\
Grader, Columbia University, Life in the Universe \hfill {2013}\\
Teaching Assistant, Columbia University, Galaxies \& Cosmology \hfill {2013}\\
Teaching Assistant, Columbia University, Stars \& Atoms \hfill {2012}\\
Instructor, UNC-Chapel Hill, Physics Help Center \hfill {2011 -- 2012}\\
Teaching Assistant, UNC-Chapel Hill, Calculus-Based Newtonian Mechanics \hfill {2011}\\

\end{rSection}

%----------------------------------------------------------------------------------------
%	STUDENTS ADVISED
%----------------------------------------------------------------------------------------

\begin{rSection}{Students Advised}
Larry Li, Columbia University, undergraduate research \hfill{2016 -- present}\\
Garrison Grogan, Columbia University, undergraduate research \hfill{2016 -- present}\\
Lowell Schudel, Columbia University, undergraduate research \hfill{2014 -- 2015}\\
\end{rSection}

%----------------------------------------------------------------------------------------
%	OUTREACH AND SERVICE
%----------------------------------------------------------------------------------------

\begin{rSection}{Selected Outreach and Service}
Instructor for \href{http://rv.astro.columbia.edu/}{Rooftop Variables}, Curtis High School, Staten Island, New York \hfill{2012 -- present}\\
Outreach Volunteer, bi-weekly community stargazing, Columbia University \hfill{2012 -- present}\\
Public Lecture, \textit{Our Magnetic Universe}, Columbia Astronomy Outreach Lecture Series \hfill{2015} \\ 
%Physics Tutor, Barnard Higher Education Opportunity Program \hfill{2013}\\
Founder, President, UNC-Chapel Hill Women in Physics \hfill{2010 -- 2012}\\
Member, Social Chair, UNC-Chapel Hill Society of Physics Students \hfill{2010 -- 2012}\\
Chapter Director, Mentor, UNC-Chapel Hill \href{http://striveforcollege.org/}{Strive For College} \hfill{2009 -- 2012}\\

\end{rSection}


%----------------------------------------------------------------------------------------
%	OTHER PUBLICATIONS
%----------------------------------------------------------------------------------------

\begin{rSection}{Other Publications}
{\textit{Closing My Eyes}}, \textbf{S. E. Clark}, personal essay, The Washington Post Magazine, May 2009
\end{rSection}

%\begin{rSection}{Technical Strengths}

%\begin{tabular}{ @{} >{\bfseries}l @{\hspace{6ex}} l }
%Computer Languages & Prolog, Haskell, AWK, Erlang, Scheme, ML \\
%Protocols \& APIs & XML, JSON, SOAP, REST \\
%Databases & MySQL, PostgreSQL, Microsoft SQL \\
%Tools & SVN, Vim, Emacs
%\end{tabular}

%\end{rSection}

%----------------------------------------------------------------------------------------
%	EXAMPLE SECTION
%----------------------------------------------------------------------------------------

%\begin{rSection}{Section Name}

%Section content\ldots

%\end{rSection}

%----------------------------------------------------------------------------------------

\end{document}
